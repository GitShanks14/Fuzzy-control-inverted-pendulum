\section{Conclusions}

Since the PID controller takes action while factoring 3 different facets of the deviation angle, it outperforms our current fuzzy controller. We will need to upgrade our fuzzy controller if we want it to compete. Overall, we need to design the Control Action vs $\phi$ curve better. The things we could do to achieve this include
\begin{enumerate}
	\item Increase the number of sets we can classify the state into
	\item Select the control action values better
	\item Consider multiple facets of the error, such as the integral and derivative as well
	\item Use a different class of membership functions
\end{enumerate}

Another thing we could do is to get the best of both worlds by implementing a form of fuzzy PID control. 

\section{Learning Outcomes}

We learnt how to model simple systems on MATLAB, be it linear or non-linear. We learnt how to stabilize unstable systems with both PID and fuzzy logic control. All three of us were new to fuzzy logic, and in general learnt a lot about how things work. Though we didn't get to implement everything that we learnt, we learnt a lot through this project. \\

Things we were looking forward to learn but couldn't due to time constraints include standard techniques for tuning PID for unstable systems, how to design fuzzy logic controllers properly for such systems, and multi-input fuzzy logic controllers for the same. Advanced topics that we only glanced at include adaptive or fuzzy PID controllers, and neural networks for the control of such systems. 